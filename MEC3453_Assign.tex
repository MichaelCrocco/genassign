\documentclass[a4paper,11pt]{article}
\usepackage[margin=2cm]{geometry}
%\usepackage[utf8]{inputenc}
\usepackage{amsmath}
\usepackage{float}
\usepackage{stanli}
\usepackage{tikz}
\usetikzlibrary{arrows.meta,arrows,positioning}
\usepackage{ifthen}
\usepackage{adjustbox}
\usepackage[gobble=auto]{pythontex}
\usepackage[nameinlink]{cleveref}

% Differential
\newcommand*{\D}{\mathrm{d}}
% scientific notation
\providecommand{\e}[1]{\ensuremath{\times~10^{#1}}} 

\begin{pycode}
	import math
	def n(num, sig=3):
		f = '%.' + str(sig) + 'g' 
		ret = f % num
		if 'e' in ret:
			ret = ret.replace('e', r'\e{')
			ret += '}'
		return ret
	
	def to_eng(num, sig=3):
		try:
			x = int(math.log10(num)//3)*3
		except (ValueError, KeyError):
			x = 0
		if x != 0:
			ret = str(n(num/10**x,sig)) + r' \times 10^{' + str(x) + '}'
		else:
			ret = str(n(num,sig))
		return ret

	def springRate(E,I,L):
	   	 return (3*E*I)/L
	    	
	def omega_n(K,m):
	    	return sqrt(K/m)
	    	
	def omega_d(K,m,zeta):
	    	return omega_n(K,m)*sqrt(1-zeta^2)
	    	
	def X_0(K,m,zeta):
	    	return 200/(m*omega_d(K,m,zeta))

\end{pycode}

% Commands for jinja2 templating
% Instead of empty braces (i.e. no command effect), here we make the variable uppercase red
% to emphasize in the template those variables to be replaced. Once rendered, this formatting
% has not effect as jinja replaces the \VAR{variable} entirely.
\usepackage{xcolor}
\newcommand*{\VAR}[1]{\textcolor{red}{\textbf{#1}}}
\newcommand*{\BLOCK}[1]{\textcolor{red}{\textbf{#1}}}

\usepackage{comment}
\newif\ifhidden
% This defines whether to show the hidden content or not.
\hiddenfalse 
\ifhidden 	% if \ hiddentrue
	\excludecomment{hidden}	% Exclude text within the "hidden" environment
\else   	% \ hiddenfalse
	\includecomment{hidden}		% Include text in the "hidden" environment
\fi

\title{CIV4280 Bridge Design \& Assessment\\Test for \VAR{FullName} (\VAR{StudentID})}
\date{Semester 1, 2020}
\author{Dr Colin Caprani\\Dr Mayer Melhem}

\begin{document}
\maketitle
%\thispagestyle{fancy}

\begin{abstract}



If an abstract or instruction is desired, it should be included here.
\end{abstract}

%{\sf\tighttoc}
\newpage

\section{Problem}

\begin{pycode}
	from testscript import *
	import random
	
	L_list = [3.5,3.75,4.0,4.25,4.5,4.75,5.0]  # m
	M_list = [-80, -30, -50]	# kNm
	I = 37.5e-6 	# m4
	J = 75e-6   	# m4
	E = 200     	# GPa
	G = 80      	# GPa
	
	# Unit conversions MNm2
	E *= 1e3
	G *= 1e3
	
	L = random.choice(L_list) 		# Member length
	M = random.choice(M_list) 		# y-axis moment load @ C
	
	grid = [L,I,J,E,G,M]
	
	EI_L = E*I/L
	GJ_L = G*J/L
	
	# Call analysis
	D = solve_D(grid)
	theta_C, theta_D, theta_F = D
	Fcd = eleF(eleK(grid),d_CD(D)*1e3)
	Fce = eleF(eleK(grid),d_CE(D)*1e3)
	Fcf = eleF(eleK(grid),d_CF(D)*1e3)
	Fgc = eleF(eleK(grid),d_GC(D)*1e3)
	Fcg = eleF(eleK(grid),d_CG(D)*1e3)
	
	bmd_scale=8
	tmd_scale=5
\end{pycode}

Consider a simple model of an airplane wing given in Fig 1. The wing is approximated as vibrating back and forth in its plan, massless compared to the missle carriage system (of mass, \emph{m}). The modulus and the moment of inertia are approximated by \emph{E} and \emph{I}, respectively, and \emph{l} is the length of the wing. The wing is modelled as a simple cantilever for the purpose of estimating the vibration resulting from the release of the missle, which is approximately the impulse function $\mathit{F\delta(t)}$.\hfill \break \break
Calculate the response and plot your results for the case of an aluminum wing with the following details.

\begin{itemize} %these should be appropritately randomized 
	\item $l = \py{int(l)}$ m
	\item $I = \pys{!{to_eng(I*1e12)}}$ mm$^4$
	\item $\zeta = \py{float(zeta)}$
	\item $E = \py{int(E/1e3)}$ GPa
	\item $m = \pys{!{to_eng(m*1e3)}}$ kg
\end{itemize}

Model \emph{F} as 2000N lasting for 0.1 s.

%\begin{figure}[ht]
%	\centering
%
%	\begin{pysub}
%	\setcoords{-25}{30}[1][1] 
%	\setaxis{2} 
%	\begin{adjustbox}{max width=\textwidth}
%	\begin{tikzpicture}[coords] 
%	
%	\dpoint{c}{0}{0}{0}; 
%	\dpoint{d}{ -!{L} }{0}{0};
%	\dpoint{e}{0}{ !{L} }{0}; 
%	\dpoint{f}{ !{L} }{0}{0};
%	\dpoint{g}{0}{-!{L} }{0}; 
%	\dpoint{z}{ -!{L} }{ -!{0.9*L} }{0}; 
%	
%	\daxis{1}{z}; 
%	
%	\foreach \startn/\endn in {c/d,c/e,c/f,c/g}
%		\dbeam{1}{\startn}{\endn};
%	
%	\support{1}{d}; \hinge{1}{d}
%	\support{3}{e}[90];
%	\support{1}{f}; \hinge{1}{f}
%	\support{3}{g}[270];
%	
%	\begin{scope}[force/.append style={color=red,>={Latex[length=0pt 5]}},
%				  normalLine/.append style={line width=1mm}]
%		%Fiendishly clever hack to convert float to pt for testing, hence to integer
%		\ifthenelse{\lengthtest{ !{abs(M)} pt > 0pt}}
%		{
%			\dload{4}{c}[90][-90][ !{L/3} ];
%			\dnotation{1}{c}{$!{n(-M)}$~kNm}[left=5mm];
%		}{}
%	\end{scope}
%	
%	\foreach \startn/\endn in {c/d,c/f}
%		\ddimensioning{xy}{\startn}{\endn}{-!{1.3*L}}[$!{L}$~m];
%	\foreach \startn/\endn in {c/e,c/g}
%		\ddimensioning{yx}{\startn}{\endn}{!{1.4*L}}[$!{L}$~m];
%	
%	\dnotation{1}{c}{$C$}[below=2mm];
%	\dnotation{1}{d}{$D$}[above right];
%	\dnotation{1}{e}{$E$}[above left];
%	\dnotation{1}{f}{$F$}[above right];
%	\dnotation{1}{g}{$G$}[below right];
%	
%	\end{tikzpicture}
%	\end{adjustbox}
%	\end{pysub}
%	\caption{Grid Problem.}
%	\label{fig:grid}
%\end{figure}


\begin{hidden}	
\clearpage
\section{Solution}

\textbf{Step 1:} Describe a cantilever system.  \textit{(figure to follow)}

\textbf{Step 2:} Sketch the impulse. \textit{(figure to follow)}

\textbf{Step 3:} Describe and define the equivalent sprung mass-damper. \textit{(figure to follow)}

\textbf{Step 4:} Define the response

\begin{equation}
	x(t) = X_0\exp^{\zeta\omega_nt}sin(\omega_dt + \phi)
\end{equation}

\begin{equation}
	X_0 = \frac{(Impulse)}{m\omega_d}
\end{equation}

...but
\begin{equation}
\begin{split}
	Impulse & = F\delta(t)\\
		& = 2000(0.1)\\
		& =200 N\cdot s
\end{split}
\end{equation}

\begin{equation}
\begin{split}

	K &\left & = \!\frac{3EI}{l^3}\right

	& = \left\frac{3\py{to_eng(E)}\py{to_eng(I)}}
				{\py{to_eng(l)}}\times 10^{-3}\right
                
	& = \py{springRate(E,I,l)}

\end{split}
\end{equation}

\begin{equation}
\begin{split}

	\omega_n &\left & = \sqrt{\frac{K}{m}}\right

	& = \left\frac{\py{to_eng(K)}}{\py{to_eng(I)}}\right
                
	& = \py{omega_n(K,m)}

\end{split}
\end{equation}

\begin{equation}
\begin{split}

	\omega_d &\left & = \omega_n\sqrt{1-\zeta^2}\right

	& = \py{omega_n(K,m)}\sqrt{1-\py{\zeta}^2}
                
	& = \py{omega_d(K,m,zeta)} s^{-1}

\end{split}
\end{equation}

\begin{equation}
\begin{split}

	X_0 & = \frac{Impulse}{m\cdot\omega_d}
	
	& = \frac{200}{\py{m}\cdot\py{omega_d(omega_n,zeta)}}
                
	& = \py{X_0(K,m,zeta)}

\end{split}
\end{equation}

Response:

\begin{equation}
%\begin{split}

	x(t) = X_0\exp^{\zeta\omega_nt}sin(\omega_dt + \phi)
	
	& = \py{X_0(K,m,zeta)}\exp^{\py{zeta}\py{omega_n(K,m)}}sin( \py{omega_d(K,m,zeta)}t )

%\end{split}
\end{equation}

%plot the response



\end{hidden}
\end{document}